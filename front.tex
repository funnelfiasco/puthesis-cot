%
%  revised  front.tex  2011-09-02  Mark Senn  https://engineering.purdue.edu/~mark
%  created  front.tex  2003-06-02  Mark Senn  https://engineering.purdue.edu/~mark
%
%  This is ``front matter'' for the thesis.
%
%  Regarding ``References'' below:
%      KEY    MEANING
%      PU     ``A Manual for the Preparation of Graduate Theses'',
%             The Graduate School, Purdue University, 1996.
%      TCMOS  The Chicago Manual of Style, Edition 14.
%      WNNCD  Webster's Ninth New Collegiate Dictionary.
%
%  Lines marked with "%%" may need to be changed.
%

  % Dedication page is optional.
  % A name and often a message in tribute to a person or cause.
  % References: PU 15, WNNCD 332.
\begin{dedication}
  This is the optional dedication.
  The dedication usually consists of a name or cause as in:\\
  Dedicated to my grandmother.
\end{dedication}

  % Acknowledgements page is optional but most theses include
  % a brief statement of apreciation or recognition of special
  % assistance.
  % Reference: PU 16.
\begin{acknowledgments}
  This is the optional acknowledgments section.
  Most theses include brief statements of appreciation or recognition of special assistance as in:

  I wish to gratefully acknowledge my thesis committee for their insightful comments and guidance and my family for their support and encouragement.
\end{acknowledgments}

  % The preface is optional.
  % References: PU 16, TCMOS 1.49, WNNCD 927.
\begin{preface}
  This is the optional preface.

  A preface includes introductory remarks here regarding reasons for undertaking this work and method of research.

  Since everyone knows you're writing this document to get your degree, don't put that here.
  If your research was done to solve a problem that came up in industry, you may want to put that here.

  If not obvious from the rest of your thesis, you may want to describe your method of research here.

  Acknowledgements go in the ``Acknowledgments'' section and do not belong here.
\end{preface}

  % The Table of Contents is required.
  % The Table of Contents will be automatically created for you
  % using information you supply in
  %     \chapter
  %     \section
  %     \subsection
  %     \subsubsection
  % commands.
  % Reference: PU 16.
\tableofcontents

  % If your thesis has tables, a list of tables is required.
  % The List of Tables will be automatically created for you using
  % information you supply in
  %     \begin{table} ... \end{table}
  % environments.
  % Reference: PU 16.
\listoftables

  % If your thesis has figures, a list of figures is required.
  % The List of Figures will be automatically created for you using
  % information you supply in
  %     \begin{figure} ... \end{figure}
  % environments.
  % Reference: PU 16.
\listoffigures

  % List of Symbols is optional.
  % Reference: PU 17.
\begin{symbols}
  $m$& mass\cr
  $v$& velocity\cr
\end{symbols}

  % List of Abbreviations is optional.
  % Reference: PU 17.
\begin{abbreviations}
  abbr& abbreviation\cr
  bcf& billion cubic feet\cr
  BMOC& big man on campus\cr
  AT& Aviation Technology\cr
  BCM& Building and Contruction Management\cr
  CGT& Computer Graphics Technology\cr
  CIT& Computer and Information Technology\cr
  CoT& College of Technology\cr
  ECET& Electrical and Computer Engineering Technology\cr
  MET& Mechanical Engineering Technology\cr
  TLI& Technology Leadership and Innovation\cr
\end{abbreviations}

  % Nomenclature is optional.
  % Reference: PU 17.
\begin{nomenclature}
  Alanine& 2-Aminopropanoic acid\cr
  Valine& 2-Amino-3-methylbutanoic acid\cr
\end{nomenclature}

  % Glossary is optional
  % Reference: PU 17.
\begin{glossary}
  chair& the person in charge of a meeting or organization\cr
  chick& female, usually young\cr
  dude& male, usually young\cr
\end{glossary}

  % Abstract is required.
  % Note that the information for the first paragraph of the output
  % doesn't need to be input here...it is put in automatically from
  % information you supplied earlier using \title, \author, \degree,
  % and \majorprof.
  % Reference: PU 17.
  %
  % The abstract is required for theses and dissertations. It is
  % optional for thesis proposals and dissertation proposals. 
\begin{abstract}
  The first paragraph must contain your name as it appears on the title page but with the last name first, the abbreviation of the degree title, the name of the institution granting the degree, the month and year the degree is awarded, the title of the thesis, and the name(s) of your major professor(s).
  This paragraph is automatically generated by \LaTeX based on the information you provided in thesis.tex.
  Follow the first paragraph with a statement of your thesis problem, a brief exposition of the research and a condensed summary of your findings.
\end{abstract}
