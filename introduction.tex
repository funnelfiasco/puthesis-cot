%   ================================================================================
%   introduction.tex
%   ================================================================================
%   College of Technology Thesis Template - 2010-05-06 - Robert S. Cutler (rcutler@purdue.edu)
%     Updated on 2012-04-26 by Jason St. John (jstjohn@purdue.edu)
%
%   Based on the LaTeX thesis template file:
%       revised  introduction.tex  2012-01-18  Mark Senn  https://engineering.purdue.edu/~mark
%       created  introduction.tex  2002-06-03  Mark Senn  https://engineering.purdue.edu/~mark
%   ================================================================================
%
%   Each chapter can be in a separate file and included in the main thesis document using an \include statement.
%
%   Within each chapter, use \section, \subsection, and \subsubsection appropriately.
%   ================================================================================
%
%  This is the introduction chapter for a simple, example thesis.
%


\chapter{Introduction}
% You should include a few sentences to introduce your thesis before transitioning
% into the scope, significance, etc. sections. This ensures that (1) the reader has
% some narrative introducing the thesis before jumping into the various sections and
% (2) that the first section heading is separated from the chapter heading with text.

This is a sample introduction that you should change to fit your thesis topic.
This is another sentence in the very beginnings of the introduction.
Note that the first paragraph after a heading is not indented.

This is a sentence.
This is a sentence.

\section{Scope}

This paper critically examines why you should write your thesis using \LaTeX in the College of Technology.

This is a sentence.


\section{Significance}

This is a significant sentence.


\section{Research Question}

What is your research question?


\section{Assumptions}

The assumptions for this study include:
\begin{itemize}
\item first
\item second
\item third
\end{itemize}


\section{Limitations}

The limitations for this study include:
\begin{itemize}
\item first
\item second
\item third
\end{itemize}


\section{Delimitations}

The delimitations for this study include:
\begin{itemize}
\item first
\item second
\item third
\end{itemize}


\section{Definitions}
In the broader context of thesis writing, we define the following terms:
\begin{italicdesc}
\item[\LaTeX:] A typesetting application that makes really impressive looking documents
\item[Purdue University:] (commonly: \textit{Purdue}) A public university founded in 1869
\item[Boilermakers:] Purdue University's official mascot
\end{italicdesc}


\section{Other Stuff}

\subsection{Thesis}

A long research paper used to satisfy requirements for a Master's Degree.


\subsection{Thesis Proposal}

The proposal for the thesis.


\subsection{Dissertation}

A longer research paper used to satisfy requirements for the Doctor of Philosophy (Ph.D.) degree.


\subsection{Dissertation Proposal}

The proposal for the dissertation.


\section{Summary}

This chapter provided the scope, significance, research question, assumptions, limitations, delimitations, definitions, and other background information for the research project.
The next chapter provides a review of the literature relevant to ``your thesis''.

