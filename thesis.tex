%   ================================================================================
%   thesis.tex
%   ================================================================================
%   College of Technology Thesis Template - 2010-05-06 - Robert S. Cutler (rcutler@purdue.edu)
%     Updated on 2012-04-26 by Jason St. John (jstjohn@purdue.edu)
%     This template has been tested using TeX Live 2012
%
%   Based on the LaTeX thesis template file:
%       thesis.tex  2011-07-01  Mark Senn  https://engineering.purdue.edu/~mark
%   ================================================================================
%
%  This is the thesis ``root file''.
%
%  To print the final copy of your thesis put a '%'
%  in front of the \includeonly command and type
%  (from page 71 of _LaTeX User's Guide and Reference Manual_, 2nd edition):
%      latex thesis
%      bibtex thesis
%      latex thesis
%      latex thesis
%
%  In "Reference:" listings below:
%      KEY  MEANING
%      TM   ``A Manual for the Preparation of Graduate Theses'',
%             seventh revised edition, The Graduate School, 2006.
%             http://www2.itap.purdue.edu/gradschool//Publications/graduate-thesis-manual.pdf
%      PU   ``A Manual for the Preparation of Graduate Theses'',
%           The Graduate School, Purdue University, 1996.
%           http://www2.itap.purdue.edu/gradschool//Publications/graduate-thesis-manual.pdf
%
%  Search for "CHANGE" below and change things as necessary.
%  I recommend putting "%%" before any existing lines that
%  need to be changed and adding your new line(s) immediately
%  below the existing lines.
%

% See https://engineering.purdue.edu/~mark/puthesis/#Options
% for documentclass options.
%     'thesisproposal' is a new type added by Robert S. Cutler
%     'dissertationproposal' is a new type added by Robert S. Cutler
%     'nochapterblankpages' removes the requirement that chapters must start on odd-numbered pages
%     'uglyheadings' is the correct style for CoT
% CHANGE NEXT LINE?
\documentclass[tech,thesisproposal,apacite,nochapterblankpages,uglyheadings]{puthesis-cot}

%   ------------------------------------------------------------------------------------------------------------------------
%   \usepackage{indentfirst}
%   
%       Ensures that the first paragraph after a \section or \subsection tag is indented
%       properly.  (All other paragraphs are automatically indented properly.)
%
%   \newcommand{\ip}{\mbox{}\indent}
%       Unfortunately, the previous package does not work with the first paragraph 
%       after a \subsubsection tag.  Put the tag \ip after a \subsubsection tag to indent the
%       first paragraph after the \subsubsection tag.
%   ------------------------------------------------------------------------------------------------------------------------
\usepackage{indentfirst}
\newcommand{\ip}{\mbox{}\indent}

% The "fixltx2e" package is required for superscripting and subscripting.
\usepackage{fixltx2e}

% Per Dr. Mohler, there should be no hyphenation throughout the thesis.
\usepackage[none]{hyphenat}
% \raggedright prevents long, non-hyphenated words from bleeding into the margins.
\raggedright
% \raggedright gets rid of normal paragraph indentation, so we must reset that here.
\parindent=0.5in

% Define "align" environment used in demo-mathematics.tex.
% CHANGE NEXT LINE?
\usepackage{amsmath}

% Define "multicols" environment environment used in demo-multicols.tex.
% CHANGE NEXT LINE?
\usepackage{multicol}

% Define "subfigure" environment used in "demo-figure.tex".
% CHANGE NEXT LINE?
\usepackage{subfigure}

% Title of thesis (used on cover and in abstract).
% The title shown must be the full, official title of the
% thesis.  Superscripts and subscripts are not permitted in
% the title.
% Reference: TM 26.
% Use \title{Put Title Here} for a one-line title.
% Use \\ to separate lines.
% Put % at the end of the last line to avoid getting an extra space
% in the abstract.
% There are two forms of title: one line or more than one line.
% There are examples of both below.
% Only use one \title.
% CHANGE NEXT FOUR LINES.
\title{An Example Thesis Done with \LaTeX}
\title{%
  An Example Thesis Done with \LaTeX\\
  with a Very Long Title in the\\
  College of Technology%
}

% First author name with first name first is used for cover.
% Second author name with last name first is used for abstract.
% Your full name as it appears in the University records appears
% on the cover.
% Reference: TM 26, 29.
% There are two forms of author, with and without initials.
% There are examples of both below.
% Only use one \author line.
%
% Template:
% \author{Firstname M. Lastname}{Lastname, Firstname M.}
%
% CHANGE NEXT TWO LINES.
\author{Mark Senn}{Senn, Mark}
\author{Mark D. Senn}{Senn, Mark D.}

% First is long title of degree (used on cover).
% Second is abbreviation for degree (used in abstract).
% Third is the month the degree was (will be) awarded (used on cover
% and abstract).
% Last is the year the degree was (will be) awarded (used on cover
% and abstract).
% The degree title for all doctoral candidates is ``Doctor of Philosophy.''
% The precise degree names for master's candidates appear in the list of
% ``Degrees Offered'' in the Graduate School bulletin.
% The date is the month and year that the degree is actually awarded.
% (If you have registered for ``degree only,'' revise the thesis title
% page to reflect the new date on which the degree is to be awarded.)
% Reference: TM 26--27, 30.
% Only use one \pudegree line.
% CHANGE NEXT LINE?
\pudegree{Doctor of Philosophy}{Ph.D.}{May}{2012}
\pudegree{Master of Science}{M.S.}{May}{2012}

% Major professor (used in abstract).
% Use, for example:
%     \majorprof{John Q. Professor}
%     \majorprofs{John Q. Professor and Thomas R. Jones}
%     \majorprofs{John Q. Professor, Thomas R. Jones, and David S. Smith}
% depending on the number of major professors you have.
% CHANGE NEXT LINE.
\majorprof{James L. Mohler}

% Campus (used only on cover)
% Use one of the following:
%     Fort Wayne
%     Hammond
%     Indianapolis
%     West Lafayette
%     Westville
% Reference: TM 27.
% CHANGE NEXT LINE?
\campus{West Lafayette}

% My command definitions not specific to my thesis.
% CHANGE NEXT LINE?
\input{mydefs}


% My command definitions specific to my thesis.

% CHANGE NEXT LINE TWO LINES?
% Set things up so \margins will show where the margins on the page are.
\newcommand{\margins}{\Repeat{Show where the margins for the page are.}{4}}

% CHANGE NEXT TWO LINES?
% Let typing "\en" be exactly the same as typing "\ensuremath". 
\let\en=\ensuremath

% CHANGE NEXT FIVE LINES?
% Define a \ve command with two arguments, so if it called with
%     \ve an
% it will expand to
%     {\en{a_1},~\en{a_2},\ \ldots,~\en{a_{n}}}
\newcommand{\ve}[2]{\en{#1_1},~\en{#1_2},\ \ldots,~\en{#1_{#2}}}


% To LaTeX only some parts of your thesis put the
% names of the parts to include here.  For example,
% \includeonly{front} would only process front.tex.
% \includeonly{front,introduction} would only process
% front.tex and introduction.tex.
% To print the final copy of your thesis put a '%'
% in front of the \includeonly command and run LaTeX
% three times to make sure that all cross-references
% are correct.  Then run BibTeX once and LaTeX twice
% more.
% CHANGE NEXT LINE?
%\includeonly{front,introduction}

\begin{document}

% Start a new volume for your thesis.  All theses must have at least one
% volume.  If your thesis is too long to fit in one binder put another
% "\volume" between chapters below.
\volume

% Front matter (dedication, etc.).
\include{front}

% Put chapter \include commands here.
% CHANGE \include{...} COMMANDS BELOW?
%   ================================================================================
%   introduction.tex
%   ================================================================================
%   College of Technology Thesis Template - 2010-05-06 - Robert S. Cutler (rcutler@purdue.edu)
%     Updated on 2012-02-05 by Jason St. John (jstjohn@purdue.edu)
%
%   Based on the LaTeX thesis template file:
%       revised  introduction.tex  2012-01-18  Mark Senn  https://engineering.purdue.edu/~mark
%       created  introduction.tex  2002-06-03  Mark Senn  https://engineering.purdue.edu/~mark
%   ================================================================================
%
%   ================================================================================
%   Each chapter can be in a separate file and included in the main thesis document using an \include statement.
%
%   Within each chapter, use \section, \subsection, and \subsubsection appropriately.
%   ================================================================================
%
%  This is the introduction chapter for a simple, example thesis.
%


\chapter{Introduction}

This is the introduction.
The first paragraph after a heading is not indented.

This is a sentence.
This is a sentence.
This is a sentence.
This is a sentence.
This is a sentence.


\section{Background and Research Question}

This paper critically examines why you should write your thesis using \LaTeX in the College of Technology.

This is a sentence.


\section{Definitions}

In the broader context of thesis writing, we define the following terms:


\subsection{Thesis}

A long research paper used to satisfy requirements for a Master's Degree.


\subsection{Thesis Proposal}

The proposal for the thesis.


\subsection{Dissertation}

A longer research paper used to satisfy requirements for the Doctor of Philosophy (Ph.D.) degree.


\subsection{Dissertation Proposal}

The proposal for the dissertation.


\subsubsection{Subsubsection heading}

This is a sentence.
This is a sentence.
This is a sentence.
This is a sentence.
This is a sentence.


% The literature review
% The placement of this section may be incorrect!
%	================================================================================
%	literaturereview.tex
%	================================================================================
%	College of Technology Thesis Template - 2010-05-06 - Robert S. Cutler (rcutler@purdue.edu)
%     Updated on 2012-02-05 by Jason St. John (jstjohn@purdue.edu)
%
%	Based on the LaTeX thesis template file:
%  		introduction.tex  2011-09-02  Mark Senn  https://engineering.purdue.edu/~mark
%	================================================================================

\chapter{Review of Relevant Literature}

\section{Section}
A section...

\subsection{Subsection}
A subsection...

\subsubsection{Subsubsection}
\ip
A subsubsection...  (Note the use of the \\ip command in the source text to handle indentation correction.)

\section{Another section}
Back to the section level...


% The methodology section
% The placement of this section may be incorrect!
%	================================================================================
%	methodology.tex
%	================================================================================
%	College of Technology Thesis Template - 2010-05-06 - Robert S. Cutler (rcutler@purdue.edu)
%     Updated on 2012-02-05 by Jason St. John (jstjohn@purdue.edu) 
%
%	Based on the LaTeX thesis template file:
%  		introduction.tex  2011-09-02  Mark Senn  https://engineering.purdue.edu/~mark
%	================================================================================

\chapter{Framework and Methodology}

Put any required text here.


%	================================================================================
%	Framework
%	================================================================================

\section{Framework}

Blah, blah, blah...


%	================================================================================
%	Methodology
%	================================================================================

\section{Methodology}
More blah, blah, blah...




% Summary and/or conclusions are optional but often used.
% The summary and/or conclusions often are the last
% major division(s) of the text.
% Reference: TM 32.
% CHANGE NEXT LINE?
%
%  summary.tex  2007-02-06  Mark Senn  https://engineering.purdue.edu/~mark
%

\chapter{Summary}

This is the summary chapter.


% Recommendations are optional.
% You may include recommendations as a major division if your
% subject matter and research dictate.
% Reference: TM 32.
% CHANGE NEXT LINE?
%
%  recommendations.tex  2007-02-06  Mark Senn  https://engineering.purdue.edu/~mark
%

\chapter{Recommendations}

Buy low.  Sell high.


% Appendices are optional.
% Appendices are not necessarily part of every thesis. Appendices are used
% for supplementary illustrative material, original data, computer programs,
% and other material not necessarily appropriate for inclusion within the
% text of your thesis. 
% Reference: TM 33.
% Use "\appendix" for one appendix or "\appendices" for more than one
% appendix.
% CHANGE NEXT 7 LINES?
\appendices
\include{demo-citations}
\include{demo-figures}
\include{demo-mathematics}
\include{demo-multicols}
\include{demo-tables}
\include{demo-text}

% Bibliography is required if you consulted any outside references.
% Reference: TM 32.
%
%  bibliography.tex  2011-07-19  Mark Senn  https://engineering.purdue.edu/~mark
%
%  This is the bibliography for the thesis.
%

% IF YOU USE BIBTEX USE THE FOLLOWING LINE:
\bibliography{all}

% If BibTeX produces a .bbl file that is close to what you want but
% not exactly right you may want to include the contents of the .bbl
% file below and use the instructions immediately below.  If you do
% this you'll want to do it after you've cited all your references
% and are running LaTeX on the final copy of your thesis.  I recommend
% running LaTeX three times after changing this file like you'd like
% it.

% OTHERWISE (YOU ARE NOT USING BIBTEX), USE THE FOLLOWING LINES:
% Format the bibliography items like your school or department wants.
%\begin{thebibliography}{}
%
%\bibitem{Kopka:2004}
%Kopka H. and Daly P. W.,
%\emph{A Guide to \LaTeX:
%    Document Preparation for Beginners and Advanced Users}.
%
%\bibitem{Lamport:1994}
%Lamport L.,
%\emph{\LaTeX: A Document Preparation System}.
%
%\bibitem{Mittelbach:2004}
%Mittelbach F. and Goossens M.,
%\emph{The \LaTeX\ Companion}.
%
%\end{thebibliography}


% Notes and footnotes are optional.
% Reference: TM 34.
% I have not implemented this yet.  Mark Senn 2002-06-03
%%\include{notes}

% A vita is optional for master's theses
% and required for doctoral dissertations.
% Reference: TM 13.
% CHANGE NEXT LINE?
%
%  vita.tex  2012-01-18  Mark Senn  https://engineering.purdue.edu/~mark
%
%  A vita is optional for masters theses
%  and required for doctoral dissertations.
%

\begin{vita}
  The remainder
  of this page was taken from page 13
  of the Grad School's thesis manual.

  Ph.D.~candidates are required to include a vita in their dissertation;
  however,
  this is optional for master's candidates.
  The vita is normally the last major division
  of the dissertation
  (unless followed by a publication)
  and should be separated from the preceding material
  by a cover sheet that is neither numbered nor counted.
  The content of this section will be largely driven
  by departmental requirements;
  in some cases,
  you may be asked to provide a curriculum vitae,
  detailing your professional and academic resume.
\end{vita}


\end{document}

% LaTeX won't read after the \end{document} command.
% You can put notes to yourself or LaTeX input not
% ready for use here if you'd like.
